\Conclusion % заключение к отчёту

В ходе проделанной работы был продемонстрирован подход для автоматизированного тестирования игр с помощью методов глубокого обучения. На примере среды из платформы \textit{Gym} было показано, что метод, награждающий агента в зависимости от нахождения ошибок, показывает лучшие результаты, нежели стандартная Deep Learning-модель.

Итоговый код реализации доступен в открытом репозитории на GitHub \cite{rl4testing}.

Несмотря на полученные результаты, вопрос автоматизированного тестирования требует дополнительных исследований. В дальнейшем следует перейти от искусственных багов к изучению ошибок, возникающих при разработке настоящих продуктов в области GameDev. Это могут быть, например, поиск областей с низкой частотой кадров в секунду (FPS) или участков, где игровой агент может застрять либо, наоборот, свободно пройти там, где это нежелательно. 

%%% Local Variables: 
%%% mode: latex
%%% TeX-master: "rpz"
%%% End: 
