
%\NirEkz{Экз. 3}                                  % Раскоментировать если не требуется
%\NirGrif{Секретно}                % Наименование грифа

%\gosttitle{Gost7-32}       % Шаблон титульной страницы, по умолчанию будет ГОСТ 7.32-2001, 
% Варианты GostRV15-110 или Gost7-32 
 
\NirOrgLongName{Правительство Российской Федерации\par
САНКТ-ПЕТЕРБУРГСКИЙ ГОСУДАРСТВЕННЫЙ УНИВЕРСИТЕТ\par
Факультет прикладной математики --- процессов управления
}                                           %% Полное название организации

%\NirUdk{УДК № 378.14}
%\NirGosNo{№ госрегистрации 01970006723}
%\NirInventarNo{Инв. № ??????}

%\NirConfirm{Согласовано}                  % Смена УТВЕРЖДАЮ
%\NirBoss[.49]{Проректор университета\\по научной работе}{И.С. Пупкин}            %% Заказчик, утверждающий НИР


%\NirReportName{Научно-технический отчет}   % Можно поменять тип отчета
%\NirAbout{О составной части \par опытно-конструкторской работы} %Можно изменить о чем отчет

%\NirPartNum{Часть}{1}                      % Часть номер

%\NirBareSubject{}                  % Убирает по теме если раскоментить

% \NirIsAnnotacion{АННОТАЦИОННЫЙ }         %% Раскомментируйте, если это аннотационный отчёт
%\NirStage{промежуточный}{Этап \No 1}{} %%% Этап НИР: {номер этапа}{вид отчёта - промежуточный или заключительный}{название этапа}
%\NirStage{}{}{} %%% Этап НИР: {номер этапа}{вид отчёта - промежуточный или 

%\Nir{АВТОМАТИЗИРОВАННОЕ ТЕСТИРОВАНИЕ С ПОМОЩЬЮ DEEP LEARNING В GAME DEV}

\NirSubject{АВТОМАТИЗИРОВАННОЕ ТЕСТИРОВАНИЕ С ПОМОЩЬЮ DEEP LEARNING В GAME DEV}                                   % Наименование темы
\NirFinal{}                        % Заключительный, если закоментировать то промежуточный
%\finalname{}               % Название финального отчета (Заключительный) 
%\NirCode{Шифр\,---\,САПР-РЛС-ФИЗТЕХ-1} % Можно задать шифр как в ГОСТ 15.110
\NirCode{}

\NirManager{Работу выполнил}{М.В. Павлов} %% Название руководителя
\NirIsp{Научный руководитель}{И.С. Блеканов} %% Название руководителя

\NirYear{2022}%% если нужно поменять год отчёта; если закомментировано, ставится текущий год
\NirTown{Санкт-Петербург}                           %% город, в котором написан отчёт
