\Introduction

В настоящее время разработка компьютерных игр (GameDev) является быстроразвивающейся индустрией. Как и любое другое программное обеспечение, игры проходят через процесс контроля качества. Так как  компьютерные игры требуют осознанного взаимодействия с некоторой средой, они плохо поддаются автоматизированному тестированию, и поэтому требуют много человеко-часов. Сокращение времени тестирования видеоигр в угоду сжатым срокам приводит к появлению программных ошибок (багов). К тому же, некоторые ошибки остаются незамеченными тестировщиками из-за «слепоты к изменениям» \cite{simons2005change}, возникающей во время контроля качества.

Исследования показывают, что баги часто являются причиной негативного пользовательского опыта (UX) \cite{kujala2011ux}, поэтому на данный момент актуален вопрос разработки методов автоматизации их поиска. С увеличением популярности моделей глубокого обучения (Deep Learning) появилось множество работ, рассматривающих их применение в разработке игр \cite{eldahshan2022deep}, в том числе "--- их тестирования \cite{tufano2022using,casamayor2022bug}.

Целью работы является создание агента, способного к автоматическому тестированию в области GameDev при помощи методов глубокого обучения. Для достижения поставленной цели необходимо решить следующие задачи:

\begin{itemize}
	\item[--] проанализировать существующие подходы к автоматизации игрового тестирования на основании глубокого обучения;
	\item[--] написать программу, демонстрирующую работу Deep Learning и его преимущества;
	\item[--] оценить результаты программы.
\end{itemize}

%
%Проверяем как у нас работают сокращения, обозначения и определения "---
%MAX, 
%\Abbrev{MAX}{Maximum ""--- максимальное значение параметра}
%API 
%\Abbrev{API}{application programming interface ""--- внешний интерфейс взаимодействия с приложением}
%с обратным прокси.
%\Define{Обратный прокси}{тип прокси-сервера, который ретранслирует}



