\chapter{Эксперимент}
\label{cha:research}

\section{Ход работы}

В данном разделе проводятся вычислительные эксперименты.

\section{Результаты эксперимента}

Результаты эксперимента приведены в табл.~\ref{tab:results}.

\begin{table}[ht]
	\centering
	\caption{Число эпизодов (из 100), в которых модели нашли хотя бы один баг.}
	\begin{tabular}{ l c c c }
		\hline
		№ & RELINE & D3QN & Случайный агент \\
		\hline
		1 & 49 & 0 & 22\\
		2 & 38 & 0 & 23\\
		3 & 37 & 0 & 21\\
		4 & 42 & 0 & 20\\
		5 & 42 & 0 & 24\\
		6 & 46 & 0 & 27\\
		7 & 49 & 0 & 26\\
		8 & 39 & 0 & 14\\
		9 & 54 & 0 & 14\\
		10 & 39 & 0 & 19\\
		\hline
		\textit{среднее} & 43,5 & 0 & 21 \\
		\textit{медиана} & 42 & 0 & 21,5 \\
		\textit{ср. кв. отклонение} & 5,72 & 0 & 4,447 \\
		\hline
	\end{tabular}
	\label{tab:results}
\end{table}

%%% Local Variables:
%%% mode: latex
%%% TeX-master: "rpz"
%%% End:
